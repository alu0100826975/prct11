\documentclass{beamer}
\usepackage[utf8]{inputenc}
\usepackage{graphicx}

\newtheorem{definicion}{Objetivo de la práctica}
%\newtheorem{ejemplo}{Ejemplo}

%%%%%%%%%%%%%%%%%%%%%%%%%%%%%%%%%%%%%%%%%%%%%%%%%%%%%%%%%%%%%%%%%%%%%%%%%%%%%%%
\title[PI]{Aproximación del número $\pi$. }
\author[Anabel]{Anabel Estévez Carrillo}
\date[25-04-2014]{25 de abril de 2014}
%%%%%%%%%%%%%%%%%%%%%%%%%%%%%%%%%%%%%%%%%%%%%%%%%%%%%%%%%%%%%%%%%%%%%%%%%%%%%%%

\usetheme{classic}

\begin{document}
  
%++++++++++++++++++++++++++++++++++++++++++++++++++++++++++++++++++++++++++++++
\begin{frame}

 \hspace*{7.0cm}

 \titlepage

 \begin{small}
   \begin{center}
    Facultad de Matemáticas \\
    Universidad de La Laguna
   \end{center}
  \end{small}

\end{frame}
%++++++++++++++++++++++++++++++++++++++++++++++++++++++++++++++++++++++++++++++ 

%++++++++++++++++++++++++++++++++++++++++++++++++++++++++++++++++++++++++++++++
\begin{frame}
 \frametitle{Índice}
 \tableofcontents[pausesections]
\end{frame}

%++++++++++++++++++++++++++++++++++++++++++++++++++++++++++++++++++++++++++++++

\section{Primera Sección}

%++++++++++++++++++++++++++++++++++++++++++++++++++++++++++++++++++++++++++++++
\begin{frame}

\frametitle{Motivación y objetivos.}

\begin{definicion}

El objetivo de este informe es exponer un programa escrito en el lenguaje de programación Python en el que se aproxime 
el valor de $\pi$. Además, se tratara de profundizar en los conocimientos adquiridos sobre \LaTeX{}, realizando un informe.

\end{definicion}

\end{frame}
%++++++++++++++++++++++++++++++++++++++++++++++++++++++++++++++++++++++++++++++ 
%++++++++++++++++++++++++++++++++++++++++++++++++++++++++++++++++++++++++++++++

\section{Segunda Sección}


%++++++++++++++++++++++++++++++++++++++++++++++++++++++++++++++++++++++++++++++
\begin{frame}

\frametitle{El número pi y su historia}

El número pi, representado por la letra griega $\pi$, equivale a la constante que relaciona el perímetro o longitud de una 
circunferencia con su diámetro. Es un número irracional y una de las constantes matemáticas más importantes.

Se estima que ya en el año 2.000 a.C. los babilonios tuvieron un acercamiento al averiguar que la circunferencia de un 
círculo suele ser poco más de tres veces el equivalente a su diámetro. Sin embargo, no fue hasta el año 225 a.C. cuando 
Arquímedes de Siracusa inició su teoría matemática. La misma se fue perfeccionando a lo largo de los siglos y en 
1706 el matemático William Jones usó por primera vez su símbolo $\pi$, aunque fue Leonhard Euler el que lo popularizó
 a partir de 1737. El número $\pi$ se utiliza en el cálculo del área del círculo, $a = \pi \times r^2$

\end{frame}
%++++++++++++++++++++++++++++++++++++++++++++++++++++++++++++++++++++++++++++++ 
%++++++++++++++++++++++++++++++++++++++++++++++++++++++++++++++++++++++++++++++

\section{Tercera Sección}

%++++++++++++++++++++++++++++++++++++++++++++++++++++++++++++++++++++++++++++++
\begin{frame}

\frametitle{Aproximaciones de pi}

El valor de $\pi$ se ha obtenido con diversas aproximaciones a lo largo de la historia. La búsqueda del mayor número de 
decimales del número $\pi$ ha supuesto un esfuerzo constante de numerosos científicos a lo largo de la historia.
Una forma exacta de poder calcular $\pi$ es la fórmula de Machin, descubierta en 1706. De este modo, 
$\pi$ se puede calcular mediante integración:

$$\int_{0}^{1} \!\frac{4}{1+x^2}\, dx = 4(atan(1) -atan(0)) = \pi $$

Por ejemplo, si utilizamos la regla del punto medio obtenemos:

\begin{center}
$ \pi \approx \frac{1}{n} \sum\limits_{i=1}^{n}f(x_i)\,$,
con $f(x) = \frac{4}{(1+x^2)}\,$,
$x_i = \frac{i - \frac{1}{2}}{n}$,
para $i = 1, \dots, n$.
\end{center}

\end{frame}

%++++++++++++++++++++++++++++++++++++++++++++++++++++++++++++++++++++++++++++++ 
%++++++++++++++++++++++++++++++++++++++++++++++++++++++++++++++++++++++++++++++

\section{Cuarta Sección}

%++++++++++++++++++++++++++++++++++++++++++++++++++++++++++++++++++++++++++++++
\begin{frame}

\frametitle{Calculando pi con python}

Vamos a escribir un programa que utilize la regla del punto medio expuesta anteriormente para calcular una aproximación de $\pi$. El recibirá
como entrada el número de subintervalos con los que se desea abordar la aproximación del número $\pi$.

\begin{block}{Salidas del programa}
  \begin{itemize}
  \item
   Los extremos de los subintervalos.
  \pause

  \item
   El punto $x_i$.
  \pause

  \item
  El valor de de la función de aproximación de $pi$, $f(x_i)$.
  \pause
  
  \item
  El resultado de la aproximación.
  \pause
  
  \item
  La constante pi con 35 decimales.

  \end{itemize}
\end{block}

\end{frame}
%++++++++++++++++++++++++++++++++++++++++++++++++++++++++++++++++++++++++++++++

\section{Bibliografía}
%++++++++++++++++++++++++++++++++++++++++++++++++++++++++++++++++++++++++++++++
\begin{frame}
  \frametitle{Bibliografía}

  \begin{thebibliography}{10}

    \beamertemplatebookbibitems
    \bibitem[Tutorial Python, 2013]{python}
    Tutorial de Python.
    (2013)
    {\small $http://docs.python.org/2/tutorial/$}

    \beamertemplatebookbibitems
    \bibitem[Tutorial Latex]{latex}
    Tutorial de Latex.
    {\small $http://latexlive.files.wordpress.com/2009/04/tablas.pdf/$}
   

  \end{thebibliography}
\end{frame}

%++++++++++++++++++++++++++++++++++++++++++++++++++++++++++++++++++++++++++++++ 


%++++++++++++++++++++++++++++++++++++++++++++++++++++++++++++++++++++++++++++++
\end{document}